\documentclass[a4paper,11pt] {article}
\usepackage[spanish]{babel}
\usepackage[utf8]{inputenc}
\usepackage{caratula}
\usepackage{a4wide}
\usepackage{graphicx}
% \usepackage{dot2texi}
% \usepackage{graphs}

\begin{document}

\titulo{Trabajo Pr\'actico}
\fecha{10/12/2009}
\materia{Sistemas Operativos}
\grupo{Nro. 10}
\integrante{Dinota, Mat\'ias}{076/07}{matiasgd@gmail.com}
\integrante{Leveroni, Luciano}{360/07}{lucianolev@gmail.com}
\integrante{Mosteiro, Agust\'in}{125/07}{agustinmosteiro@gmail.com}

\maketitle

\bigskip
\section*{Aclaraciones generales}

Antes de comenzar el an\'alisis de cada ejercicio, cabe mencionar lo siguiente: 

\section{Comandos básicos de Unix}

\begin{enumerate}
	\item \begin{enumerate}
		\item El directorio actual pasa a ser \textbf{/usr/bin}
		\item El directorio actual pasa a ser \textbf{/home/tpsisop}
		\item Al no ingresar ningún parámetro, el comando \textit{cd} cambia el directorio actual al directorio personal del usuario actual.
	\end{enumerate}
	\item El contenido del archivo es:
	\begin{verbatim}
# ~/.profile: executed by the command interpreter for login shells.
# This file is not read by bash(1), if ~/.bash_profile or ~/.bash_login
# exists.
# see /usr/share/doc/bash/examples/startup-files for examples.
# the files are located in the bash-doc package.

# the default umask is set in /etc/profile
#umask 022

# if running bash
if [ -n "$BASH_VERSION" ]; then
    # include .bashrc if it exists
    if [ -f "$HOME/.bashrc" ]; then
	. "$HOME/.bashrc"
    fi
fi

# set PATH so it includes user's private bin if it exists
if [ -d "$HOME/bin" ] ; then
    PATH="$HOME/bin:$PATH"
fi
	\end{verbatim}
	\item El único archivo encontrado es \textbf{/vmlinuz}
	\setcounter{enumi}{8}
	\item 
	\begin{verbatim}
	 127.0.0.1	localhost
::1     ip6-localhost ip6-loopback
	\end{verbatim}
	\item La nueva password es guiguigui
\end{enumerate}


\end{document}