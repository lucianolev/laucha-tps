\documentclass[a4paper,11pt] {article}
\usepackage[spanish]{babel}
\usepackage[utf8]{inputenc}
\usepackage{caratula}
\usepackage{a4wide}
%\usepackage{graphicx}
% \usepackage{dot2texi}
% \usepackage{graphs}

\begin{document}

\titulo{Trabajo Pr\'actico Nro. 1}
\fecha{21/04/2010}
\materia{Ingenier\'ia de Software I}
\grupo{}
\integrante{Dinota, Mat\'ias}{076/07}{matiasgd@gmail.com}
\integrante{Frid, Igal Pablo}{231/07}{ipfrid@gmail.com}
\integrante{Huel, Federico Ariel}{329/07}{federico.huel@gmail.com}
\integrante{Leveroni, Luciano}{360/07}{lucianolev@gmail.com}
\integrante{Mosteiro, Agust\'in}{125/07}{agustinmosteiro@gmail.com}

\maketitle

\bigskip
\section*{Aclaraciones generales}

%Antes de comenzar el an\'alisis de los algoritmos, cabe mencionar lo siguiente:
%
%\begin{itemize}
% \item La implementaci\'on de los todos algoritmos se realiz\'o en \textbf{lenguaje Java}, haciendo uso de las librer\'ias est\'andar del mismo.
% \item Para el c\'alculo de tiempo de los algoritmos se utiliz\'o la funci\'on \textbf{nanoTime()} de la clase System de Java. Con el fin de aumentar la precisi\'on de las mediciones, se utiliz\'o el comando \textbf{nice} para darle m\'axima prioridad a la tarea.
% \item El c\'odigo fuente de los algoritmos aqu\'i analizados se encuentran en el archivo \textit{ResolvedorCIPM.java}.
% \item El c\'odigo fuente de los programas encargados de hacer uso de los algoritmos y necesarios para compilar la aplicaciones son: MainCIPM.java, ResolvedorCIPM.java, GrafoNPonderados.java, PesoNodoComparator.java, LectorDeGrafos.java y EscritorDeSoluciones.java.
% \item Para la lectura y escritura de los datos se utilizaron clases provistas por el lenguaje Java. No se har\'a referencia a estos algoritmos ya que no resultan de inter\'es para el trabajo aqu\'i presentado.
% \item Los gr\'aficos se realizaron con \textbf{GNUPlot} y las tablas con OpenOffice Calc. En los casos considerados pertinentes, se utiliz\'o una escala logar\'itmica con el fin de poder visualizar mejor los resultados.
%\end{itemize}
%
%Con respecto al análisis de resultados y a las pruebas realizadas vale hacer la siguiente aclaración. Los grafos generados aletoriamente poseerán nodos con pesos comprendidos entre $10$ y cantidad de nodos del grafo por $10$. Las adyacencias de los nodos es generada también aleatoriamente de modo que el grado de cada nodo sea un número entre 0 y (n-1)*k, con $k$ un parámetro que llamaremos \textit{densidad} de aquí en adelante, donde $0 < k \leq 1$. De este modo, los experimentos mostrarán el comportamiento de los algoritmos para grafos ``promedio'' (sin particularidades) con una cierta cantidad de nodos y de cierta densidad.

\section*{Introducci\'on}

El objetivo del siguiente trabajo es modelar un sistema para una cadena de pizzer\'ias, denominada Pizza Hack. Con este fin, se utilizar\'an t\'ecnicas de ingenier\'ia de requerimientos, a saber: Diagrama de Contexto y de Objetivos. A su vez, se incluyen en el presente trabajo las explicaciones y detalles de cada uno de los diagramas para clarificar lo expuesto de forma gr\'afica en dichos diagramas. 

El sistema a modelar pretende resolver las necesidades y problemas de una cadena de pizzer\'ias t\'ipica, en la que cada uno de los locales atiende al p\'ublico a trav\'es de un mostrador, es decir, no ofrece la posibilidad de consumir las pizzas en el mismo local. Adem\'as, cabe aclarar que la cadena de pizzer\'ias desea mantener el mismo men\'u en todos sus locales, por lo que un cambio en el men\'u de un local deber\'a verse reflejado en la cartilla de los restantes. Tambi\'en se debe mencionar que los locales no ofrecen servicio de delivery y, en caso de que alg\'un pedido no se pueda preparar en el local donde se encuentra el cliente, se ofrecer\'a derivar su pedido a otro local de la cadena. Por \'ultimo, hay que tener en cuenta que el sistema no debe tener un servidor centralizado para la coordinaci\'on entre locales.

Por medio de la aplicaci\'on de las t\'ecnicas de inginier\'ia de requerimientos se busca explorar distintas opciones para modelar el sistema deseado por la cadena de pizzas. Finalmente, se estudiar\'an las ventajas y desventajas que conlleva elegir cada una de estas alternativas en los aspectos que consideramos relevantes para el funcionamiento del sistema y el beneficio de la cadena de pizzas.  

\section*{Presunciones}

Para la confección del trabajo, se hicieron las siguientes presunciones:

\begin{itemize}
    \item Pizza Hack no brinda servicio de delivery ni tampoco brinda la posibilidad de comer en el local.
    \item La máquina no contempla manejo de dinero.

\end{itemize}

\section*{Vistas}

\section*{Escenarios}
A continuaci\'on se mostrar\'an algunos de los posibles escenarios que se podr\'ian llegar a presentarse en la cadena de pizzas describida. 
\begin{itemize}
	\item Un cliente realiza un pedido en un local y el mismo puede ser preparado. Una vez listo el pedido, se le entrega al cliente y este se retira satisfecho.
	\item Un cliente realiza un pedido en un local y el mismo no puede ser preparado. Se le ofrece al cliente la opci\'on de derivar su pedido al local m\'as cercano y este acepta, retirando posteriormente el pedido en el otro local.
    \item Al pizzero se le cae una de las pizzas al suelo. En ese instante, el pizzero informa al encargado lo ocurrido y el mismo da de baja el stock correspondiente. A continuación, el pizzero vuelve a realizar la pizza.
    \item Al pizzero se le cae una de las pizzas al suelo. Debido a la falta de stock, el pizzero no puede preparar la pizza nuevamente por lo que no puede cumplir con el pedido.
    \item Un cliente realiza un pedido que no se puede realizar en el local debido a la falta de stock. Además, al haber una falla en la red de comunicación entre locales, tampoco es posible ofrecer retirar el pedido en otro local.
    \item Luego de preparar una pizza, el stock de uno de los ingredientes necesarios para preparar alguna pizza popular se está agotando por lo que el sistema informa al repositor que debe comprar dicho ingrediente de inmediato.
    \item Luego de realizar el pedido, el cliente se arrepiente y decide cancelarlo. En este caso, la cajera ingresa dicha cancelación en el sistema. Si la pizza ya fue preparada, los ingredientes se pierden.
    \item Se realiza una derivación de un pedido a otro local pero el cliente que lo había efectuado no se presenta en el mismo pasada las 3 hrs. Al igual que en el escenario anterior, los ingredientes se pierden.
\end{itemize}

\section*{Discusión}

\section*{Conclusiones Generales}

%El trabajo aqui realizado se encargó de mostrar diversos algoritmos para resolver el problema del conjunto independiente de peso máximo para un grafo de nodos ponderados cualquiera. En primer lugar se realizó un simple algoritmo exacto de orden $O(2^n)$ para posteriores comparaciones con las heuristicas propuestas. Este orden exponencial, mostró luego que el algoritmo sólo termina en un tiempo razonable para grafos relativamente pequeños (hasta 36 nodos, con baja densidad). 
%
%Luego, se comenzó por mostrar distintas heurísticas constructivas para generar soluciones al problema para grafos más grandes. Como vimos, si bien se mostraron heurísticas mejores que otras, todas ellas resultaron estar relativamente lejos de la solución exacta. Por pruebas realizadas, se concluyó que la mejor heurística resultó ser una heurística golosa que añade los nodos con mayor relación peso/grado del grafo a la solucion parcial. De este modo, el algorimto encontraría soluciones en un tiempo polinomial $O(n*m)$ permitiendo ser ejecutado sobre grafos de un orden de $10^3$ de baja densidad en tiempos razonables.
%
%En la siguiente etapa se estudiaron 2 heurísticas de búsqueda local con el fin de mejorar las soluciones contructivas obtenidas. Resultó interesante ver como una busqueda local bastante más inteligente dió resultados notablemente mejores, aún siendo de una complejidad temporal aceptable como ser $O(n*m)$. Más aún, realizar la heurística sola o utilizar una busqueda local a posteriori conserva la complejidad de $O(n*m)$ por lo cual se puede concluir que la heurística por sí sola carece de utilidad. Otro punto interesante, es que la búsqueda realiza una cantidad de iteraciónes que aumenta muy levemente con respecto al tamaño del grafo, por lo cual fue posible acotar dicho parámetro sin perder en calidad de soluciones encontradas.
%
%Por último se estudió una versión del esquema metaheurístico GRASP que muestra como la mejora en la calidad de las soluciones es aún mayor que realizar una búsqueda local. Tal como explica la bibliografía asociada, este resultado tiene relación con el hecho de que se realizan distintas busquedas locales en diversas soluciones iniciales lo que resulta en el aumentar notablemente la posibilidad de hallar soluciones mejores. Además, al haber fijado la cantidad de iteraciones del procedimiento, la complejidad teórica sigue siendo del mismo órden que realizar una única busqueda local, es decir $O(n*m)$. Esto último muestra como este esquema metaheurístico resulta de gran utilidad al momento de querer encontrar soluciones de calidad para problemas de gran tamaño imposibles de resolver de forma exacta.

\section*{Referencias}
%\begin{itemize}
% \item ``Greedy Randomized Adaptive Search Procedures'' - Thomas A. Feo, Mauricio G.C. Resende
% \item Art\'iculo de Wikipedia sobre Conjunto independiente
%\end{itemize}

\end{document}