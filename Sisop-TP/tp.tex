\documentclass[a4paper,11pt] {article}
\usepackage[spanish]{babel}
\usepackage[utf8]{inputenc}
\usepackage{caratula}
\usepackage{a4wide}
\usepackage{graphicx}
% \usepackage{dot2texi}
% \usepackage{graphs}

\begin{document}

\titulo{Trabajo Pr\'actico}
\fecha{10/12/2009}
\materia{Sistemas Operativos}
\grupo{Nro. 10}
\integrante{Dinota, Mat\'ias}{076/07}{matiasgd@gmail.com}
\integrante{Leveroni, Luciano}{360/07}{lucianolev@gmail.com}
\integrante{Mosteiro, Agust\'in}{125/07}{agustinmosteiro@gmail.com}

\maketitle

\bigskip
\section*{Aclaraciones generales}

Antes de comenzar el an\'alisis de cada ejercicio, cabe mencionar lo siguiente: 

\section{Comandos básicos de Unix}

\begin{enumerate}
	\item \begin{enumerate}
		\item El directorio actual pasa a ser \textbf{/usr/bin}.
		\item El directorio actual pasa a ser \textbf{/home/tpsisop}.
		\item Al no ingresar ningún parámetro, el comando \textit{cd} cambia el directorio actual al directorio personal del usuario actual.
	\end{enumerate}
	\item El contenido del archivo es:
	\begin{verbatim}
# ~/.profile: executed by the command interpreter for login shells.
# This file is not read by bash(1), if ~/.bash_profile or ~/.bash_login
# exists.
# see /usr/share/doc/bash/examples/startup-files for examples.
# the files are located in the bash-doc package.

# the default umask is set in /etc/profile
#umask 022

# if running bash
if [ -n "$BASH_VERSION" ]; then
    # include .bashrc if it exists
    if [ -f "$HOME/.bashrc" ]; then
	. "$HOME/.bashrc"
    fi
fi

# set PATH so it includes user's private bin if it exists
if [ -d "$HOME/bin" ] ; then
    PATH="$HOME/bin:$PATH"
fi
	\end{verbatim}
	\item El único archivo encontrado es \textbf{/vmlinuz}.
	\item Se generó el directorio utilizando el comando \textbf{mkdir /home/tpsisop/tp}.
	\item Se copió el archivo utilizando el comando \textbf{cp /etc/passwd /home/tpsisop/tp/}.
	\item Se cambió el grupo utilizando el comando \textbf{chgrp tpsisop /home/tpsisop/tp/passwd}.
	\item Se cambió el usuario utilizando el comando \textbf{chown tpsisop /home/tpsisop/tp/passwd}.
	\item Se cambiaron los permiso del archivo \textbf{/home/tpsisop/tp/passwd}.
		\begin{itemize}
			\item Para que el propietario tenga permisos de lectura, escritura y ejecución: \textbf{chmod u$+$rwx /home/tpsisop/tp/passwd}.
			\item Para que el grupo tenga sólo permisos de lectura y ejecución: \textbf{chmod g$-$w$+$rx /home/tpsisop/tp/passwd}.
			\item Para que el resto tenga sólo permisos de ejecución: \textbf{chmod o$-$rw$+$x /home/tpsisop/tp/passwd}.
		\end{itemize}
	\item 
		\begin{itemize}
		 \item 	
			\begin{verbatim}
	 			127.0.0.1	localhost
			::1     ip6-localhost ip6-loopback
			\end{verbatim}
		 \item 	
		\end{itemize}
	\item La nueva password es \textit{guiguigui} con el comando \textbf{passwd}.
	\item Se borró el archivo con el comando \textbf{rm /home/tpsisop/tp/passwd}.
	\item Se enlazaron los archivos con los comandos \textbf{ln /etc/passwd /tmp/contra1}, \textbf{ln /etc/passwd /tmp/contra2} y \textbf{ln -s /etc/passwd /tmp/contra3} respectivamente.
	\item Se montó el CD-ROM de instalacion de Ubuntu JeOS con el comando \textbf{mount /dev/sr0/ /home/tpsisop/tp/}. El contenido del directorio es:
	\begin{verbatim}
		-r-xr-xr-x 1 root root   942 2008-04-22 03:07 cdromupgrade
		dr-xr-xr-x 3 root root  2048 2009-07-14 13:23 dists
		dr-xr-xr-x 3 root root  2048 2009-07-14 13:23 doc
		dr-xr-xr-x 3 root root  2048 2009-07-14 13:24 install
		dr-xr-xr-x 2 root root 12288 2009-07-14 13:24 isolinux
		-r--r--r-- 1 root root 47110 2009-07-14 13:24 md5sum.txt
		dr-xr-xr-x 2 root root  2048 2009-07-14 13:23 pics
		dr-xr-xr-x 4 root root  2048 2009-07-14 13:23 pool
		dr-xr-xr-x 2 root root  2048 2009-07-14 13:23 preseed
		-r--r--r-- 1 root root   228 2009-07-14 13:23 README.diskdefines
		lr-xr-xr-x 1 root root     1 2009-07-14 13:23 ubuntu -> .
	\end{verbatim}
	Se utilizó el comando \textbf{mount} para mostrar los siguientes \textit{filesystems} montados:
	\begin{verbatim}
		/dev/sda1 on / type ext3 (rw,relatime,errors=remount-ro)
		proc on /proc type proc (rw,noexec,nosuid,nodev)
		/sys on /sys type sysfs (rw,noexec,nosuid,nodev)
		varrun on /var/run type tmpfs (rw,noexec,nosuid,nodev,mode=0755)
		varlock on /var/lock type tmpfs (rw,noexec,nosuid,nodev,mode=1777)
		udev on /dev type tmpfs (rw,mode=0755)
		devshm on /dev/shm type tmpfs (rw)
		devpts on /dev/pts type devpts (rw,gid=5,mode=620)
		/dev/scd0 on /home/tpsisop/tp type iso9660 (ro)
	\end{verbatim}
	\item Se utilizó el comando \textbf{df} para mostrar el espacio libre de los \textit{filesystems} montados:
	\begin{verbatim}
		Filesystem            Size  Used Avail Use% Mounted on
		/dev/sda1             494M  423M   46M  91% /
		varrun                252M   32K  252M   1% /var/run
		varlock               252M     0  252M   0% /var/lock
		udev                  252M   40K  252M   1% /dev
		devshm                252M     0  252M   0% /dev/shm
		/dev/scd0             101M  101M     0 100% /home/tpsisop/tp
	\end{verbatim}
	\item TODO
	\item Se desmont\'o el CD-ROM de instalaci\'on de Ubuntu JeOS con el comando \textbf{unmount /dev/scd0}.
	\item La maquina virtual lleva 58 minutos de ejecuci\'on. Esto se corrobor\'o con el comando \textbf{uptime}.
	\item La versi\'on de kernel utilizada es la 2.6.24-24-virtual. Esta informaci\'on se obtuvo utilizando el comando \textbf{uname -a}.

\end{enumerate}

\section{Comandos Extendidos de Unix}

\begin{enumerate}
	\item Para escribir HOLA en la pantalla cada vez que se loguee un usuario se agreg\'o la siguiente l\'inea al archivo 	/etc/bash.bashrc:
	\textbf{echo "HOLA"}
	
	Para escribir BUENOS DIAS en la pantalla cada vez que se encienda la m\'aquina se agreg\'o la siguiente l\'inea al 			archivo /etc/rc.local:
	\textbf{echo "BUENOS DIAS"}
	
	Para escribir ADIOS en la pantalla cada vez que se desloguee un usuario se cre\'o un archivo llamado logout en el 			directorio /etc. El contenido del archivo es el siguiente:
	\textbf{echo "ADIOS"}
	Luego, se agreg\'o en el archivo /etc/rc.local la siguiente l\'inea:
	\textbf{trap '/etc/logout;exit' 0}
	
	Para escribir HASTA LA VISTA BABY en la pantalla cada vez que se apague la m\'aquina se agreg\'o la siguiente l\'inea 	al archivo /etc/rc0.d:
	\textbf{echo "HASTA LA VISTA BABY"}
	
	\item Para montar una imagen de floppy se utiliz\'o el siguiente comando:
		\textbf{mount -o loop imagen.img /media/floppy1/}
		
		Para montar una imagen iso se utiliz\'o el siguiente comando:
		\textbf{mount -o loop imagen.iso /media/iso/}
		
	\item Se agreg\'o un alias en el archivo .bashrc sin modificar su fecha (timestamp). Para esto se utiliz\'o el comando \textbf{touch -t timestamp}. El timestamp del archivo previo a la modificaci\'on se obtuvo con el comando  \textbf{ls -l}.
	El d\'ia y la hora del sistema se modific\'o con el comando \textbf{date timestamp}.
	
	\item TODO
	\item
		\begin{enumerate}
			\item Se guard\'o la informaci\'on en el archivo \textit{config} utilizando el comando \textbf{ls -Rl /etc > /home/tpsisop/tp/config}.
			\item El archivo \textit{config} posee 789 l\'ineas, 5061 palabras y 39070 caracteres. Se obtuvo la informaci\'on mediante el comando \textbf{wc config}.
			\item Se agrego el contenido ordenado del archivo \textit{/etc/passwd} al final del archivo \textit{config} con el comando \textbf{sort /etc/passwd $>>$ /home/tpsisop/tp/config}.
			\item El archivo \textit{config} posee 813 l\'ineas, 5090 palabras y 40051 caracteres. Se obtuvo la informaci\'on mediante el comando \textbf{wc config}.
			\item Se realiz\'o lo pedido ejecutando el siguiente comando: \textbf{ls -l /usr/bin/a* | grep apt | wc}.
		\end{enumerate}
	
\end{enumerate}

\section*{Temas del Sistema Operativo}

\begin{enumerate}
  \item \textbf{File System}
    Los \textit{hardlinks} apuntan a una estructura llamada i-nodo que contiene informaci��n sobre el archivo al que hace referencia el link y punteros a los bloques de memoria f��sica en donde est�� alojado dicho archivo. Cada i-nodo almacena en uno de sus campos la cantidad de links que existen al archivo al que referencia. De este modo, s��lo se elimina un i-nodo, y su correspondiente archivo, cuando la cantidad de links al archivo es 0, es decir, si se elimina un \textit{hardlink}, pero siguen existiendo links al archivo, este no ser�� eliminado.
    Un i-nodo contiene la siguiente informaci��n:
    \begin{itemize}
      \item Modo (tipo de archivo y permisos)
      \item Cantidad de links
      \item UID del owner
      \item GID del owner
      \item Tama�0�9o del archivo (en bytes)
      \item Fecha en la que el archivo fue accedido por ��ltima vez
      \item Fecha en la que el archivo fue modificado por ��ltima vez
      \item Fecha en la que el i-nodo fue modificado por ��ltima vez
      \item 12 punteros a bloques
      \item 1 puntero indirecto a bloques
      \item 1 puntero doble indirecto a bloques
      \item 1 puntero triple indirecto a bloques
      \item Estado del i-nodo (flags)
      \item Cantidad de bloques que ocupa el archivo
      \item Campos extra o reservados
    \end{itemize}

\end{enumerate}



\end{document}